% The Slide Definitions
%document
\documentclass[10pt]{beamer}
%theme
% \usetheme{metropolis}
% packages
\usepackage{color}
\usepackage{listings}
\usepackage[ngerman]{babel}
\usepackage[utf8]{inputenc}
\usepackage{multicol}


% color definitions
\definecolor{mygreen}{rgb}{0,0.6,0}
\definecolor{mygray}{rgb}{0.5,0.5,0.5}
\definecolor{mymauve}{rgb}{0.58,0,0.82}

\lstset{
    backgroundcolor=\color{white},
    % choose the background color;
    % you must add \usepackage{color} or \usepackage{xcolor}
    basicstyle=\footnotesize\ttfamily,
    % the size of the fonts that are used for the code
    breakatwhitespace=false,
    % sets if automatic breaks should only happen at whitespace
    breaklines=true,                 % sets automatic line breaking
    captionpos=b,                    % sets the caption-position to bottom
    commentstyle=\color{mygreen},    % comment style
    % deletekeywords={...},
    % if you want to delete keywords from the given language
    extendedchars=true,
    % lets you use non-ASCII characters;
    % for 8-bits encodings only, does not work with UTF-8
    frame=single,                    % adds a frame around the code
    keepspaces=true,
    % keeps spaces in text,
    % useful for keeping indentation of code
    % (possibly needs columns=flexible)
    keywordstyle=\color{blue},       % keyword style
    % morekeywords={*,...},
    % if you want to add more keywords to the set
    numbers=left,
    % where to put the line-numbers; possible values are (none, left, right)
    numbersep=5pt,
    % how far the line-numbers are from the code
    numberstyle=\tiny\color{mygray},
    % the style that is used for the line-numbers
    rulecolor=\color{black},
    % if not set, the frame-color may be changed on line-breaks
    % within not-black text (e.g. comments (green here))
    stepnumber=1,
    % the step between two line-numbers.
    % If it's 1, each line will be numbered
    stringstyle=\color{mymauve},     % string literal style
    tabsize=4,                       % sets default tabsize to 4 spaces
    % show the filename of files included with \lstinputlisting;
    % also try caption instead of title
    language = Python,
	showspaces = false,
	showtabs = false,
	showstringspaces = false,
	escapechar = ,
}

\def\ContinueLineNumber{\lstset{firstnumber=last}}
\def\StartLineAt#1{\lstset{firstnumber=#1}}
\let\numberLineAt\StartLineAt



\newcommand{\codeline}[1]{
	\alert{\texttt{#1}}
}


% Author and Course information
% This Document contains the information about this course.

% Authors of the slides
\author{Felix Döring, Felix Wittwer}

% Name of the Course
\institute{Python-Kurs}

% Fancy Logo 
\titlegraphic{\hfill\includegraphics[height=1.25cm]{../templates/fsr_logo_cropped}}



% Custom Bindings
% \newcommand{\codeline}[1]{
%	\alert{\texttt{#1}}
%}

\usepackage{comment}
% Presentation title
\title{Decorators}
\date{\today}


\begin{document}

\maketitle

\begin{frame}{Gliederung}
		\setbeamertemplate{section in toc}[sections numbered]
		\tableofcontents
\end{frame}


\section{Fakten über Funktionen}
\begin{frame}{Fakten über Funktionen}
	\begin{itemize}
		\item Funktionen können Variablen zugewiesen werden,
		\lstinputlisting[lastline=8]{resources/06_decorators/functions.py}
		\item Sie können in Funktionen definiert werden,
		\lstinputlisting[firstline=11, lastline=19]{resources/06_decorators/functions.py}
	\end{itemize}
\end{frame}
\begin{frame}{Fakten über Funktionen}
	\begin{itemize}
		\item Sie können andere Funktionen zurückgeben,
		\lstinputlisting[firstline=22, lastline=29]{resources/06_decorators/functions.py}
	\end{itemize}
\end{frame}
\begin{frame}{Fakten über Funktionen}
	\begin{itemize}
		\item Sie können als Parameter mitgegeben werden
		\lstinputlisting[firstline=32]{resources/06_decorators/functions.py}
	\end{itemize}
\end{frame}

\section{Decorator}
\subsection{einfache Decorator}
\begin{frame}{einfache Decorator}
	Decorator sind Wrapper über existierende Funktionen. Dabei werden die zuvor genannten Eigenschaften verwendet.\\
	Eine Funktion, die eine weitere als Argument hat, erstellt eine neue Funktion.
	\lstinputlisting[lastline=9]{resources/06_decorators/decorator.py}
\end{frame}
\begin{frame}{einfache Decorator}
	\lstinputlisting[firstline=11,lastline=18]{resources/06_decorators/decorator.py}
\end{frame}

\begin{frame}[fragile]{einfache Decorator}
	Durch das \alert{\texttt{@} Symbol} lässt sich der Decorator wesentlich einfacher verwenden.
	\lstinputlisting[firstline=22, lastline=35]{resources/06_decorators/decorator.py}
	Es können auch mehrere Decorator übereinander geschrieben werden.
\end{frame}

\subsection{Decorator mit Argumenten}
\begin{frame}{Decorator mit Argumenten}
	Decorator erwarten Funktionen als Argumente. Aus diesem Grund kann man nicht einfach andere Argumente mitgeben, sondern man muss eine Funktion schreiben, die dann den Decorator erstellt.
	\lstinputlisting[firstline= 38]{resources/06_decorators/decorator.py}
\end{frame}
	
% nothing to do from here on
\end{document}
