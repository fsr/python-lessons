%%	Getting started:
%%	1) Copy this file and name the copy it like the topic it covers
%%	2) In the following make sure to modify the copy, NOT THE ORIGINAL
%%	3) Look for the points marked with "TODO" and complete them
%%	4) compile this file

%% Nothing to modify here.
%% make sure to include this before anything else

\documentclass{beamer}
%\usetheme{Szeged}

% packages
\usepackage{color}
\usepackage{listings}

% color definitions
\definecolor{mygreen}{rgb}{0,0.6,0}
\definecolor{mygray}{rgb}{0.5,0.5,0.5}
\definecolor{mymauve}{rgb}{0.58,0,0.82}

% this is needed since the tudbeamer messes things up
\setbeamercolor{title}{fg=white}
\setbeamercolor{subtitle}{fg=white}
\setbeamercolor{supertitle}{fg=white}
\setbeamerfont{supertitle}{series=\bfseries,family=\sffamily}


% re-format the title frame page
\makeatletter
\def\supertitle#1{\gdef\@supertitle{#1}}%
\setbeamertemplate{title page}
{
  \vbox{}
  \vfill
  \begin{centering}
  \begin{beamercolorbox}[sep=8pt,center]{title}
      \usebeamerfont{supertitle}\@supertitle
   \end{beamercolorbox}
    \begin{beamercolorbox}[sep=8pt,center]{title}
    	\usebeamerfont{title}
    	\inserttitle\par%
      	\ifx\insertsubtitle\@empty%
     	\else%
        \vskip0.25em%
        {\usebeamerfont{subtitle}\usebeamercolor[fg]{subtitle}\insertsubtitle\par}%
      	\fi%
    \end{beamercolorbox}%
    \vskip1em\par
    \begin{beamercolorbox}[sep=8pt,center]{author}
      \usebeamerfont{author}\insertauthor
    \end{beamercolorbox}
    \begin{beamercolorbox}[sep=8pt,center]{institute}
      \usebeamerfont{institute}\insertinstitute
    \end{beamercolorbox}
   \begin{beamercolorbox}[sep=8pt,center]{date}
      \usebeamerfont{date}\insertdate
    \end{beamercolorbox}\vskip0.5em
    {\usebeamercolor[fg]{titlegraphic}\inserttitlegraphic\par}
  \end{centering}
  \vfill
}
\makeatother


% insert frame number
%\expandafter\def\expandafter\insertshorttitle\expandafter{%
%      \insertshorttitle\hfill%
%\insertframenumber\,/\,\inserttotalframenumber}

% preset-listing options
\lstset{
  backgroundcolor=\color{white},
  % choose the background color;
  % you must add \usepackage{color} or \usepackage{xcolor}
  basicstyle=\footnotesize,
  % the size of the fonts that are used for the code
  breakatwhitespace=false,
  % sets if automatic breaks should only happen at whitespace
  breaklines=true,                 % sets automatic line breaking
  captionpos=b,                    % sets the caption-position to bottom
  commentstyle=\color{mygreen},    % comment style
  % deletekeywords={...},
  % if you want to delete keywords from the given language
  extendedchars=true,
  % lets you use non-ASCII characters;
  % for 8-bits encodings only, does not work with UTF-8
  frame=single,                    % adds a frame around the code
  keepspaces=true,
  % keeps spaces in text,
  % useful for keeping indentation of code
  % (possibly needs columns=flexible)
  keywordstyle=\color{blue},       % keyword style
  % morekeywords={*,...},
  % if you want to add more keywords to the set
  numbers=left,
  % where to put the line-numbers; possible values are (none, left, right)
  numbersep=5pt,
  % how far the line-numbers are from the code
  numberstyle=\tiny\color{mygray},
  % the style that is used for the line-numbers
  rulecolor=\color{black},
  % if not set, the frame-color may be changed on line-breaks
  % within not-black text (e.g. comments (green here))
  stepnumber=1,
  % the step between two line-numbers.
  % If it's 1, each line will be numbered
  stringstyle=\color{mymauve},     % string literal style
  tabsize=4,                       % sets default tabsize to 4 spaces
  title=\lstname
  % show the filename of files included with \lstinputlisting;
  % also try caption instead of title
}

% macro for code inclusion
\newcommand{\includecode}[2][c]{
	\lstinputlisting[caption=#2, style=custom#1]{#2}
}
	% nothing to do here
% TODO change "course_info" to the name of your actual …_info(.tex)
%% Fill in metadata here that do not change over the course
%% They all are marked with the term "TODO". 
%% Search functions usually do the trick

% TODO select the targeted language
% Select neither when using tudbeamer
%\usepackage[english]{babel}
% \usepackage[ngerman]{babel}

% TODO select the encoding
\usepackage[utf8]{inputenc}
% usepackage[latin1]{inputenc}
\usepackage[T1]{fontenc}

\newcommand{\course}{
	Einführung in Python
}

\author{
	Justus Adam, Felix D\"oring
}

\lstset{
	% TODO adapt these settings to your mainly used language
	% also see http://en.wikibooks.org/wiki/LaTeX/Source_Code_Listings
	% NOTE you can override these settings in individual cases 
	language = Python,
	showspaces = false,
	showtabs = false,
	showstringspaces = false,
	escapechar = 
}

%% User defined macros here

% Does not work in tables! You have to use \lstinline$...$ instead!
\newcommand{\codeline}[1]{\colorbox{codegray}{\lstinline$#1$}}

% define my own colors
\definecolor{codegray}{gray}{0.97}
\definecolor{stringgreen}{rgb}{0.0, 0.7, 0.6} % TODO modify this if you have not already done so

% meta-information
\newcommand{\topic}{
	% TODO fill in the actual topic
	Funktionen (fortgeschritten)
}

% nothing to do here
\title{\topic}
\supertitle{\course}
\date{\today}

% the actual document
\begin{document}

\maketitle

\begin{frame}
	\tableofcontents
\end{frame}


\section{Nutzung von Funktionen}
\begin{frame}[fragile]{Funktionen als Werte}
  Funktionen k\"onnen wie alle anderen Werte zugewiesen werden\\
  \lstinputlisting[lastline=5]{resources/07_funtions_advanced/use_func.py}

  \ \\[.25cm]
  Oder als Parameter mitgegeben werden\\
  \lstinputlisting[firstline=8, lastline=12]{resources/07_funtions_advanced/use_func.py}
\end{frame}

\begin{frame}[fragile]{Methoden sind Funktionen}
  \lstinputlisting[firstline=15]{resources/07_funtions_advanced/use_func.py}
\end{frame}


\subsection{Default Parameter}
\begin{frame}[fragile]{Default Parameter}
  \begin{itemize}
  	\item Funktionen k\"onnen vordefinierte Werte für Parameter haben.
  	\item Parameter ohne \texttt{default}-Werten werden positionale Argumente genannt
  	\item Parameter mit \texttt{default}-Werten werden name
  \end{itemize}
  \lstinputlisting[lastline=5]{resources/07_funtions_advanced/default.py}
\end{frame}

\begin{frame}{ACHTUNG!}
	\center{\textbf{Niemals \texttt{mutable Values} (änderbare Werte) als \texttt{default}-Parameter verwenden!}}\\[1cm]
	\begin{description}
		\item[mutable Values] \ \ \ \texttt{list}, \texttt{dict}, \texttt{set} und eigene Klassen \\
\ \ \ \ \ (bzw. deren Attribute)
		\item[immutable Values] \texttt{string}, \texttt{function}, \texttt{int}, \texttt{type} und \texttt{None}
	\end{description}
\end{frame}

\begin{frame}{ACHTUNG!}
	\center{\textbf{Warum?\\}}
	\lstinputlisting[firstline=8, lastline=10]{resources/07_funtions_advanced/default.py}
	Man denkt, die Funktion hat jedes mal eine leere List, jedoch passiert folgendes:\\
	\lstinputlisting[firstline=12, lastline=14]{resources/07_funtions_advanced/default.py}
	Die Liste wird einmalig zum Start angelegt und fortgeführt.
\end{frame}

\begin{frame}[fragile]{ACHTUNG!}
	So sollte man es machen:\\[.5cm]
	\lstinputlisting[firstline=16, lastline=19]{resources/07_funtions_advanced/default.py}
	\ \\[.5cm]
	Den \texttt{default}-Parameter als \texttt{None} setzten und dann, falls er \texttt{None} ist als z.B. leere Liste setzen.
\end{frame}


\subsection{Aufruf mit Namen}
\begin{frame}[fragile]{Aufruf mit Namen}
	Funktionsparameter können direkt mit ihrem Namen aufgerufen werden, dann spielt die Aufrufreihenfolge keine Rolle mehr.\\[1cm]
	\lstinputlisting[firstline=22]{resources/07_funtions_advanced/default.py}
\end{frame}

\begin{frame}{Aufruf mit Namen}
	Es gelten folgende Regeln:
	\begin{itemize}
		\item Alle Parameter können an ihrer Position angesprochen werden
		\item Es können auch alle mit ihrem Namen angesprochen werden
		\item Wenn eins mit dem Namen angesprochen wurde, m\"ussen die folgenden ebenfalls mit Namen angesprochen werden
	\end{itemize}
\end{frame}

% ------------------------------- Aggregatoren -------------------------------
\section{Aggregatoren}
\begin{frame}{Aggregatoren}
  \textbf{Aggregatoren} (auch Sammler genannt), sind sehr n\"utzlich, wenn man,
  zus\"atzlich zu bereits definierten Parametern, in einer Funktion eine unbestimmte
  Anzahl an Funktionsargumenten entgegen nehmen will.
\end{frame}


\subsection{Positionale Aggregatoren}
\begin{frame}[fragile]{Positionale Aggregatoren}
  \begin{itemize}
    \item jede Funktion kann einen Aggregator haben
    \item dieser muss der \textit{letzte} positionale Parameter sein
    \item Positionale Aggregatoren werden durch einen \texttt{*} gekennzeichnet
    \item nach einem Aggregator k\"onnen nur noch benamte Parameter definiert werden,
    diese m\"ussen auch mit Namen aufgerufen werden \\[.5cm]
  \end{itemize}
  Der Inhalt des Aggregators wird in einem \textit{Tupel} gespeichert:
  \lstinputlisting[firstline=1, lastline=2]{resources/07_funtions_advanced/positional_aggregators.py}
\end{frame}

\begin{frame}{Positionale Aggregatoren - Beispiel}
  \lstinputlisting[firstline=5]{resources/07_funtions_advanced/positional_aggregators.py}
\end{frame}

\begin{frame}[fragile]{Positionale Aggregatoren}
  \begin{itemize}
    \item Eine Funktion kann auch nur einen Aggregator als Parameter entgegennehmen (keine anderen Parameter)
    \item Ohne Argumente ergibt sich \texttt{len(args)} zu 0
    \item Werden keine anderen Parameter erwartet, nennt man den Aggregator meist
    \texttt{args} (kurz für \textit{Arguments}) \\[.75cm]
  \end{itemize}
  In \textbf{Python 3} kann man Aggregatoren auch ohne Namen definieren:
  \lstinputlisting{resources/07_funtions_advanced/nameless_aggr.py}
  Auf diesen Aggregator kann nicht zugegriffen werden. Er erzwingt lediglich, dass
  alle folgenden Parameter mit Namen aufgerufen werden.
\end{frame}


\subsection{Benannte Aggregatoren}
\begin{frame}[fragile]{Benannte Aggregatoren}
  Analog zu Parametern gibt es auch benannte Aggregatoren. Diese werden mit \texttt{**}
  vor dem Parameternamen definiert. Diese Aggregatoren akzeptieren lediglich
  benannte Parameter und sind vom Typ \texttt{dict}.
\end{frame}

\begin{frame}[fragile]{Benannte Aggregatoren - Beispiel}
  \lstinputlisting{resources/07_funtions_advanced/named_aggr.py} \ \\[.75cm]
  Wenn eine Funktion keine anderen Parameter erwartet, nennt man den Aggregator meist
  \texttt{**kwargs} (kurz für \textit{Keyword Arguments})
\end{frame}


\subsection{Benannte und Positionale Aggregatoren}
\begin{frame}{Benannte und Positionale Aggregatoren}
  Beide Aggregatoren können gleichzeitig in einer Funktion verwendet werden.
  Die Regel dabei ist: Von jeder Sorte nur \textit{ein} Aggregator. \\[.5cm]
  Ein Beispiel:
  \lstinputlisting[firstline=12, lastline=18]{resources/07_funtions_advanced/mixed_aggr.py}
\end{frame}

\begin{frame}{Generelle Funktionsstruktur}
  Wenn beide Aggregatoren zum Einsatz kommen sollen, ergibt sich folgende Funktionsstruktur: \\[.25cm]
  \lstinputlisting[firstline=1, lastline=9]{resources/07_funtions_advanced/mixed_aggr.py}
  \textit{Eckige Klammern stehen für optionale Parameter/Namen.}
\end{frame}

% nothing to do from here on
\end{document}
