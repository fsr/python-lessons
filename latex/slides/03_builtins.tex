%% Nothing to modify here.
%% make sure to include this before anything else

\documentclass{beamer}
%\usetheme{Szeged}

% packages
\usepackage{color}
\usepackage{listings}

% color definitions
\definecolor{mygreen}{rgb}{0,0.6,0}
\definecolor{mygray}{rgb}{0.5,0.5,0.5}
\definecolor{mymauve}{rgb}{0.58,0,0.82}

% this is needed since the tudbeamer messes things up
\setbeamercolor{title}{fg=white}
\setbeamercolor{subtitle}{fg=white}
\setbeamercolor{supertitle}{fg=white}
\setbeamerfont{supertitle}{series=\bfseries,family=\sffamily}


% re-format the title frame page
\makeatletter
\def\supertitle#1{\gdef\@supertitle{#1}}%
\setbeamertemplate{title page}
{
  \vbox{}
  \vfill
  \begin{centering}
  \begin{beamercolorbox}[sep=8pt,center]{title}
      \usebeamerfont{supertitle}\@supertitle
   \end{beamercolorbox}
    \begin{beamercolorbox}[sep=8pt,center]{title}
    	\usebeamerfont{title}
    	\inserttitle\par%
      	\ifx\insertsubtitle\@empty%
     	\else%
        \vskip0.25em%
        {\usebeamerfont{subtitle}\usebeamercolor[fg]{subtitle}\insertsubtitle\par}%
      	\fi%
    \end{beamercolorbox}%
    \vskip1em\par
    \begin{beamercolorbox}[sep=8pt,center]{author}
      \usebeamerfont{author}\insertauthor
    \end{beamercolorbox}
    \begin{beamercolorbox}[sep=8pt,center]{institute}
      \usebeamerfont{institute}\insertinstitute
    \end{beamercolorbox}
   \begin{beamercolorbox}[sep=8pt,center]{date}
      \usebeamerfont{date}\insertdate
    \end{beamercolorbox}\vskip0.5em
    {\usebeamercolor[fg]{titlegraphic}\inserttitlegraphic\par}
  \end{centering}
  \vfill
}
\makeatother


% insert frame number
%\expandafter\def\expandafter\insertshorttitle\expandafter{%
%      \insertshorttitle\hfill%
%\insertframenumber\,/\,\inserttotalframenumber}

% preset-listing options
\lstset{
  backgroundcolor=\color{white},
  % choose the background color;
  % you must add \usepackage{color} or \usepackage{xcolor}
  basicstyle=\footnotesize,
  % the size of the fonts that are used for the code
  breakatwhitespace=false,
  % sets if automatic breaks should only happen at whitespace
  breaklines=true,                 % sets automatic line breaking
  captionpos=b,                    % sets the caption-position to bottom
  commentstyle=\color{mygreen},    % comment style
  % deletekeywords={...},
  % if you want to delete keywords from the given language
  extendedchars=true,
  % lets you use non-ASCII characters;
  % for 8-bits encodings only, does not work with UTF-8
  frame=single,                    % adds a frame around the code
  keepspaces=true,
  % keeps spaces in text,
  % useful for keeping indentation of code
  % (possibly needs columns=flexible)
  keywordstyle=\color{blue},       % keyword style
  % morekeywords={*,...},
  % if you want to add more keywords to the set
  numbers=left,
  % where to put the line-numbers; possible values are (none, left, right)
  numbersep=5pt,
  % how far the line-numbers are from the code
  numberstyle=\tiny\color{mygray},
  % the style that is used for the line-numbers
  rulecolor=\color{black},
  % if not set, the frame-color may be changed on line-breaks
  % within not-black text (e.g. comments (green here))
  stepnumber=1,
  % the step between two line-numbers.
  % If it's 1, each line will be numbered
  stringstyle=\color{mymauve},     % string literal style
  tabsize=4,                       % sets default tabsize to 4 spaces
  title=\lstname
  % show the filename of files included with \lstinputlisting;
  % also try caption instead of title
}

% macro for code inclusion
\newcommand{\includecode}[2][c]{
	\lstinputlisting[caption=#2, style=custom#1]{#2}
}
	% nothing to do here
% TODO change "course_info" to the name of your actual …_info(.tex)
%% Fill in metadata here that do not change over the course
%% They all are marked with the term "TODO". 
%% Search functions usually do the trick

% TODO select the targeted language
% Select neither when using tudbeamer
%\usepackage[english]{babel}
% \usepackage[ngerman]{babel}

% TODO select the encoding
\usepackage[utf8]{inputenc}
% usepackage[latin1]{inputenc}
\usepackage[T1]{fontenc}

\newcommand{\course}{
	Einführung in Python
}

\author{
	Justus Adam, Felix D\"oring
}

\lstset{
	% TODO adapt these settings to your mainly used language
	% also see http://en.wikibooks.org/wiki/LaTeX/Source_Code_Listings
	% NOTE you can override these settings in individual cases 
	language = Python,
	showspaces = false,
	showtabs = false,
	showstringspaces = false,
	escapechar = 
}

%% User defined macros here

% Does not work in tables! You have to use \lstinline$...$ instead!
\newcommand{\codeline}[1]{\colorbox{codegray}{\lstinline$#1$}}

% define my own colors
\definecolor{codegray}{gray}{0.97}
\definecolor{stringgreen}{rgb}{0.0, 0.7, 0.6} % TODO modify this if you have not already done so
\usepackage[utf8]{inputenc}

% meta-information
\newcommand{\topic}{
	% TODO fill in the actual topic
	Builtin Datenstrukturen
}

% nothing to do here
\title{\topic}
\supertitle{\course}
\date{\today}

% the actual document
\begin{document}

\maketitle

\begin{frame}
	\tableofcontents
\end{frame}

\section{exceptions}
\begin{frame}{Exception Handling}
\begin{itemize}
	\item Alle Exceptions erben von \texttt{Exception}
	\item Catching mit try/except
	\item \texttt{finally} um Code auszuführen, der \textit{unbedingt} laufen muss, egal ob eine Exception vorliegt oder nicht
\end{itemize}
\end{frame}
\begin{frame}{Exception Handling}
\lstinputlisting{resources/03_builtins/exceptions.py}
\end{frame}

\section{booleans}
\begin{frame}{Boolsche Werte}
\begin{itemize}
	\item \textit{type} ist \texttt{bool}
	\item Mögliche Werte: \texttt{True} oder \texttt{False}
	\item Operationen sind \textit{und}, \textit{oder}, \textit{nicht} \texttt{and, or, not}
\end{itemize}
\end{frame}

\section{list}
\begin{frame}{list}

\begin{itemize}
	\item enthält variable Anzahl von Objekten
	\item eine Liste kann beliebig viele verschiedene Datentypen enthalten (z.B. \texttt{bool} und \texttt{list})
	\item Auch Listen können in Listen gespeichert werden!
	\item Listenobjekte haben eine feste Reihenfolge (\textit{first in, last out})
	\item optimiert für einseitige Benutzung wie z.B. Queue (\texttt{append} und \texttt{pop})
\end{itemize}
\end{frame}
\begin{frame}{list}
	\lstinputlisting{resources/03_builtins/list.py}
\end{frame}

\section{tuple}
\begin{frame}{tuple}
\begin{itemize}
	\item Gruppiert Daten
	\item kann nicht mehr verändert werden, sobald es erstellt wurde
	\item Funktionen mit mehreren Rückgabewerten geben ein Tupel zurück
\end{itemize}
\end{frame}
\begin{frame}{tuple}
	\lstinputlisting{resources/03_builtins/tuple.py}
\end{frame}


\section{dict}
\begin{frame}{dict}
\begin{itemize}
	\item einfache HashMap
	\item ungeordnet
	\item jeder hashbare Typ kann ein Key sein
	\item jedem Key ist dann ein Value zugeordnet
\end{itemize}
\end{frame}
\begin{frame}{dict}
\lstinputlisting{resources/03_builtins/dict.py}
\end{frame}

\section{set/frozenset}
\begin{frame}{set/frozenset}
\begin{itemize}
	\item kann nur hashbare Einträge enthalten
	\item \texttt{set} selbst ist nicht hashbar
	\item \texttt{frozensets} sind hashbar, jedoch nicht mehr veränderbar
	\item enhält jedes Element nur einmal
	\item schnellere Überprüfung mit \texttt{in} (prüft, ob Element enthalten ist)
	% TODO
	\item Mögliche Operationen: \texttt{superset()}, \texttt{subset()}, \texttt{isdisjoint()}, \texttt{difference()}, \texttt{<, >, disdisjoint(), -}
	\item ungeordnet
	\item (frozen)sets können frozensets enthalten
\end{itemize}
\end{frame}
\begin{frame}{set/frozenset}
	\lstinputlisting{resources/03_builtins/set.py}
\end{frame}

\section{iteraton}
\begin{frame}{Iteration}
\begin{itemize}
	\item nur foreach
	\item für Iterationen über Integer gibt es \texttt{range([start], stop, step=1)}
	\item um Iteratoren zu kombinieren kann man \texttt{zip(iterator_1, iterator_2, ..., iterator_n)} verwenden
	\item alles mit einer \texttt{__iter__} Methode ist iterierbar
	% TODO: stateful Iterator erklären oder entfernen
	\item \texttt{iter(iterable)} konstruiert einen \textit{stateful iterator}
\end{itemize}
\end{frame}
\begin{frame}{Iteration}
	% TODO am Ende noch mal gucken, ob die Zeilen passen!
	\lstinputlisting[lastline=28]{resources/03_builtins/iterate.py}
\end{frame}
\begin{frame}{Iteration}
	\lstinputlisting[firstline=30]{resources/03_builtins/iterate.py}
\end{frame}

\section{unpacking}
\begin{frame}{Unpacking}
\begin{itemize}
	\item einfaches Auflösen von Listen und Tupeln in einzelne Variablen
	\item nützlich in \textt{for} Schleifen
\end{itemize}
\end{frame}
\begin{frame}{unpacking}
	\lstinputlisting{resources/03_builtins/unpacking.py}
\end{frame}

\section{Contenxt Manager}
\begin{frame}{Context Manager}
\begin{itemize}
	\item Aufruf mit \texttt{with}
	\item kann jedes Objekt sein, welches eine \texttt{__enter__} und \texttt{__exit__} Methode hat
	\item praktisch beim \textit{File Handling}
\end{itemize}
\end{frame}
\begin{frame}{Context Manager}
	\lstinputlisting{resources/03_builtins/cm.py}
\end{frame}

\section{File Handling}
\begin{frame}{File Handling}
\begin{itemize}
	\item Dateien können mit \texttt{open(filename, mode="r")} geöffnet werden
	\item \textit{File Handler} sind Iteratoren über die Zeilen einer Datei
	\item \textbf{Wichtig:} File Handler müssen auch wieder geschlossen werden
	\item \texttt{r} steht für Lesezugriff,  \texttt{w} für Schreibzugriff
\end{itemize}[.5cm]
\textbf{Beachte:} Wird eine Datei mit Schreibzugriff geöffnet, wird sie geleert! Also wichtige Inhalte vorher auslesen.
\end{frame}
\begin{frame}{File Handling}
	\lstinputlisting{resources/03_builtins/file.py}
\end{frame}

% nothing to do from here on
\end{document}
