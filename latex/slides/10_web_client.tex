\documentclass[ignorenonframetext,]{beamer}
\setbeamertemplate{caption}[numbered]
\setbeamertemplate{caption label separator}{:}
\setbeamercolor{caption name}{fg=normal text.fg}
\usepackage{amssymb,amsmath}
\usepackage{ifxetex,ifluatex}
\usepackage{fixltx2e} % provides \textsubscript
\usepackage{lmodern}
\ifxetex
  \usepackage{fontspec,xltxtra,xunicode}
  \defaultfontfeatures{Mapping=tex-text,Scale=MatchLowercase}
  \newcommand{\euro}{€}
\else
  \ifluatex
    \usepackage{fontspec}
    \defaultfontfeatures{Mapping=tex-text,Scale=MatchLowercase}
    \newcommand{\euro}{€}
  \else
    \usepackage[T1]{fontenc}
    \usepackage[utf8]{inputenc}
      \fi
\fi
% use upquote if available, for straight quotes in verbatim environments
\IfFileExists{upquote.sty}{\usepackage{upquote}}{}
% use microtype if available
\IfFileExists{microtype.sty}{\usepackage{microtype}}{}
\usepackage{color}
\usepackage{fancyvrb}
\newcommand{\VerbBar}{|}
\newcommand{\VERB}{\Verb[commandchars=\\\{\}]}
\DefineVerbatimEnvironment{Highlighting}{Verbatim}{commandchars=\\\{\}}
% Add ',fontsize=\small' for more characters per line
\newenvironment{Shaded}{}{}
\newcommand{\KeywordTok}[1]{\textcolor[rgb]{0.00,0.44,0.13}{\textbf{{#1}}}}
\newcommand{\DataTypeTok}[1]{\textcolor[rgb]{0.56,0.13,0.00}{{#1}}}
\newcommand{\DecValTok}[1]{\textcolor[rgb]{0.25,0.63,0.44}{{#1}}}
\newcommand{\BaseNTok}[1]{\textcolor[rgb]{0.25,0.63,0.44}{{#1}}}
\newcommand{\FloatTok}[1]{\textcolor[rgb]{0.25,0.63,0.44}{{#1}}}
\newcommand{\ConstantTok}[1]{\textcolor[rgb]{0.53,0.00,0.00}{{#1}}}
\newcommand{\CharTok}[1]{\textcolor[rgb]{0.25,0.44,0.63}{{#1}}}
\newcommand{\SpecialCharTok}[1]{\textcolor[rgb]{0.25,0.44,0.63}{{#1}}}
\newcommand{\StringTok}[1]{\textcolor[rgb]{0.25,0.44,0.63}{{#1}}}
\newcommand{\VerbatimStringTok}[1]{\textcolor[rgb]{0.25,0.44,0.63}{{#1}}}
\newcommand{\SpecialStringTok}[1]{\textcolor[rgb]{0.73,0.40,0.53}{{#1}}}
\newcommand{\ImportTok}[1]{{#1}}
\newcommand{\CommentTok}[1]{\textcolor[rgb]{0.38,0.63,0.69}{\textit{{#1}}}}
\newcommand{\DocumentationTok}[1]{\textcolor[rgb]{0.73,0.13,0.13}{\textit{{#1}}}}
\newcommand{\AnnotationTok}[1]{\textcolor[rgb]{0.38,0.63,0.69}{\textbf{\textit{{#1}}}}}
\newcommand{\CommentVarTok}[1]{\textcolor[rgb]{0.38,0.63,0.69}{\textbf{\textit{{#1}}}}}
\newcommand{\OtherTok}[1]{\textcolor[rgb]{0.00,0.44,0.13}{{#1}}}
\newcommand{\FunctionTok}[1]{\textcolor[rgb]{0.02,0.16,0.49}{{#1}}}
\newcommand{\VariableTok}[1]{\textcolor[rgb]{0.10,0.09,0.49}{{#1}}}
\newcommand{\ControlFlowTok}[1]{\textcolor[rgb]{0.00,0.44,0.13}{\textbf{{#1}}}}
\newcommand{\OperatorTok}[1]{\textcolor[rgb]{0.40,0.40,0.40}{{#1}}}
\newcommand{\BuiltInTok}[1]{{#1}}
\newcommand{\ExtensionTok}[1]{{#1}}
\newcommand{\PreprocessorTok}[1]{\textcolor[rgb]{0.74,0.48,0.00}{{#1}}}
\newcommand{\AttributeTok}[1]{\textcolor[rgb]{0.49,0.56,0.16}{{#1}}}
\newcommand{\RegionMarkerTok}[1]{{#1}}
\newcommand{\InformationTok}[1]{\textcolor[rgb]{0.38,0.63,0.69}{\textbf{\textit{{#1}}}}}
\newcommand{\WarningTok}[1]{\textcolor[rgb]{0.38,0.63,0.69}{\textbf{\textit{{#1}}}}}
\newcommand{\AlertTok}[1]{\textcolor[rgb]{1.00,0.00,0.00}{\textbf{{#1}}}}
\newcommand{\ErrorTok}[1]{\textcolor[rgb]{1.00,0.00,0.00}{\textbf{{#1}}}}
\newcommand{\NormalTok}[1]{{#1}}

% Comment these out if you don't want a slide with just the
% part/section/subsection/subsubsection title:
\AtBeginPart{
  \let\insertpartnumber\relax
  \let\partname\relax
  \frame{\partpage}
}
\AtBeginSection{
  \let\insertsectionnumber\relax
  \let\sectionname\relax
  \frame{\sectionpage}
}
\AtBeginSubsection{
  \let\insertsubsectionnumber\relax
  \let\subsectionname\relax
  \frame{\subsectionpage}
}

\setlength{\parindent}{0pt}
\setlength{\parskip}{6pt plus 2pt minus 1pt}
\setlength{\emergencystretch}{3em}  % prevent overfull lines
\providecommand{\tightlist}{%
  \setlength{\itemsep}{0pt}\setlength{\parskip}{0pt}}
\setcounter{secnumdepth}{0}

\title{Python als Webclient}
\date{}

\begin{document}
\frame{\titlepage}

\section{urllib.request}\label{urllib.request}

\begin{frame}

\begin{quote}
The urllib.request module defines functions and classes which help in
opening URLs (mostly HTTP) in a complex world --- basic and digest
authentication, redirections, cookies and more.
\end{quote}

\end{frame}

\begin{frame}[fragile]{urlopen}

\ldots{} öffnet eine URL. (surprise)

\begin{verbatim}
urlopen(url, data=None, [timeout, ]\*, cafile=None, capath=None, cadefault=False, context=None)
\end{verbatim}

\textbf{url} kann ein String sein, wie `http://python.org' oder
`ftp://store.stuff.de', oder eine komplexere Anfrage in Form eines
Request objects sein.\\
\textbf{data} enthält Daten, die an den Server gesendet werden. Muss vom
Typ \texttt{bytes} oder ein iterable von \texttt{bytes} Objekten sein.

Bei URLs mit \emph{http}-Requests wird ein httplib.client.HTTPResponse
Objekt zurück gegeben.\\
Bei \emph{ftp}, \emph{file} und \emph{data} ein
urllib.response.addinfourl Objekt.

\end{frame}

\begin{frame}[fragile]{Request Klasse}

\begin{verbatim}
Request(url, data=None, headers={}, origin_req_host=None, unverifiable=False, method=None)
\end{verbatim}

\textbf{url} muss ein String mit einer gültigen URL sein\\
\textbf{data} wie bei urlopen \textbf{headers} entweder ein
\texttt{dict} mit \{ Header-Name : Header-Value, \ldots{} \} oder eine
\texttt{list} von Tupeln mit {[}( Header-Name, Header-Value ),
\ldots{}{]}\\
\textbf{method} ein String, welcher die Art des HTTP Request angibt
(\texttt{HEAD}, \texttt{GET}, \texttt{POST},\ldots{})

\end{frame}

\begin{frame}[fragile]

Die Request Klasse wird verwendet, wenn man z.B. die gesendeten Header,
wie `Content-Type' oder `User-Agent' oder die Method, wie `POST', `PUT'
oder `HEAD' kontrollieren möchte.

\begin{Shaded}
\begin{Highlighting}[]
\NormalTok{r }\OperatorTok{=} \NormalTok{Request(}
    \StringTok{'http://python.org'}\NormalTok{,}
    \NormalTok{headers}\OperatorTok{=}\NormalTok{\{ }\StringTok{'content-type'}\NormalTok{: }\StringTok{'application/json'} \NormalTok{\},}
    \NormalTok{method}\OperatorTok{=}\StringTok{'PUT'}
\NormalTok{)}
\end{Highlighting}
\end{Shaded}

\end{frame}

\begin{frame}[fragile]{HTTPResponse Klasse}

Die Instanzen der HTTPResponse Klasse werden nicht vom Nutzer erstellt.

\begin{verbatim}
http.client.HTTPResponse(sock, debuglevel=0, method=None, url=None)
\end{verbatim}

Sie beinhalten Funktionen wie \texttt{read()}, \texttt{getheader()} oder
\texttt{getheaders()} und Variablen wie \texttt{status} oder
\texttt{version}.

\texttt{read()} gibt zurückgelieferten Ihnalt aus, \texttt{getheader()}
und \texttt{getheaders()} einen oder alle Header

\texttt{status} gibt den HTML Statuscode an \texttt{version} die HTML
version

\end{frame}

\section{Requests}\label{requests}

\begin{frame}[fragile]

Das \href{https://github.com/kennethreitz/requests}{Requests} Modul ist
eine gute Alternative zu \texttt{urllib.request}. Es vereinfacht HTTP
Requests.

\begin{Shaded}
\begin{Highlighting}[]
\OperatorTok{>>>} \NormalTok{r }\OperatorTok{=} \NormalTok{requests.get(}\StringTok{'https://api.github.com'}\NormalTok{, auth}\OperatorTok{=}\NormalTok{(}\StringTok{'user'}\NormalTok{, }\StringTok{'pass'}\NormalTok{))}
\OperatorTok{>>>} \NormalTok{r.status_code}
\DecValTok{204}
\OperatorTok{>>>} \NormalTok{r.headers[}\StringTok{'content-type'}\NormalTok{]}
\CommentTok{'application/json'}
\end{Highlighting}
\end{Shaded}

\end{frame}

\end{document}
