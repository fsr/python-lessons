
%%	Getting started:
%%	1) Copy this file and name the copy it like the topic it covers
%%	2) In the following make sure to modify the copy, NOT THE ORIGINAL
%%	3) Look for the points marked with "TODO" and complete them
%%	4) compile this file

%% Nothing to modify here.
%% make sure to include this before anything else

\documentclass{beamer}
%\usetheme{Szeged}

% packages
\usepackage{color}
\usepackage{listings}

% color definitions
\definecolor{mygreen}{rgb}{0,0.6,0}
\definecolor{mygray}{rgb}{0.5,0.5,0.5}
\definecolor{mymauve}{rgb}{0.58,0,0.82}

% this is needed since the tudbeamer messes things up
\setbeamercolor{title}{fg=white}
\setbeamercolor{subtitle}{fg=white}
\setbeamercolor{supertitle}{fg=white}
\setbeamerfont{supertitle}{series=\bfseries,family=\sffamily}


% re-format the title frame page
\makeatletter
\def\supertitle#1{\gdef\@supertitle{#1}}%
\setbeamertemplate{title page}
{
  \vbox{}
  \vfill
  \begin{centering}
  \begin{beamercolorbox}[sep=8pt,center]{title}
      \usebeamerfont{supertitle}\@supertitle
   \end{beamercolorbox}
    \begin{beamercolorbox}[sep=8pt,center]{title}
    	\usebeamerfont{title}
    	\inserttitle\par%
      	\ifx\insertsubtitle\@empty%
     	\else%
        \vskip0.25em%
        {\usebeamerfont{subtitle}\usebeamercolor[fg]{subtitle}\insertsubtitle\par}%
      	\fi%
    \end{beamercolorbox}%
    \vskip1em\par
    \begin{beamercolorbox}[sep=8pt,center]{author}
      \usebeamerfont{author}\insertauthor
    \end{beamercolorbox}
    \begin{beamercolorbox}[sep=8pt,center]{institute}
      \usebeamerfont{institute}\insertinstitute
    \end{beamercolorbox}
   \begin{beamercolorbox}[sep=8pt,center]{date}
      \usebeamerfont{date}\insertdate
    \end{beamercolorbox}\vskip0.5em
    {\usebeamercolor[fg]{titlegraphic}\inserttitlegraphic\par}
  \end{centering}
  \vfill
}
\makeatother


% insert frame number
%\expandafter\def\expandafter\insertshorttitle\expandafter{%
%      \insertshorttitle\hfill%
%\insertframenumber\,/\,\inserttotalframenumber}

% preset-listing options
\lstset{
  backgroundcolor=\color{white},
  % choose the background color;
  % you must add \usepackage{color} or \usepackage{xcolor}
  basicstyle=\footnotesize,
  % the size of the fonts that are used for the code
  breakatwhitespace=false,
  % sets if automatic breaks should only happen at whitespace
  breaklines=true,                 % sets automatic line breaking
  captionpos=b,                    % sets the caption-position to bottom
  commentstyle=\color{mygreen},    % comment style
  % deletekeywords={...},
  % if you want to delete keywords from the given language
  extendedchars=true,
  % lets you use non-ASCII characters;
  % for 8-bits encodings only, does not work with UTF-8
  frame=single,                    % adds a frame around the code
  keepspaces=true,
  % keeps spaces in text,
  % useful for keeping indentation of code
  % (possibly needs columns=flexible)
  keywordstyle=\color{blue},       % keyword style
  % morekeywords={*,...},
  % if you want to add more keywords to the set
  numbers=left,
  % where to put the line-numbers; possible values are (none, left, right)
  numbersep=5pt,
  % how far the line-numbers are from the code
  numberstyle=\tiny\color{mygray},
  % the style that is used for the line-numbers
  rulecolor=\color{black},
  % if not set, the frame-color may be changed on line-breaks
  % within not-black text (e.g. comments (green here))
  stepnumber=1,
  % the step between two line-numbers.
  % If it's 1, each line will be numbered
  stringstyle=\color{mymauve},     % string literal style
  tabsize=4,                       % sets default tabsize to 4 spaces
  title=\lstname
  % show the filename of files included with \lstinputlisting;
  % also try caption instead of title
}

% macro for code inclusion
\newcommand{\includecode}[2][c]{
	\lstinputlisting[caption=#2, style=custom#1]{#2}
}
	% nothing to do here
% TODO change "course_info" to the name of your actual …_info(.tex)
%% Fill in metadata here that do not change over the course
%% They all are marked with the term "TODO". 
%% Search functions usually do the trick

% TODO select the targeted language
% Select neither when using tudbeamer
%\usepackage[english]{babel}
% \usepackage[ngerman]{babel}

% TODO select the encoding
\usepackage[utf8]{inputenc}
% usepackage[latin1]{inputenc}
\usepackage[T1]{fontenc}

\newcommand{\course}{
	Einführung in Python
}

\author{
	Justus Adam, Felix D\"oring
}

\lstset{
	% TODO adapt these settings to your mainly used language
	% also see http://en.wikibooks.org/wiki/LaTeX/Source_Code_Listings
	% NOTE you can override these settings in individual cases 
	language = Python,
	showspaces = false,
	showtabs = false,
	showstringspaces = false,
	escapechar = 
}

%% User defined macros here

% Does not work in tables! You have to use \lstinline$...$ instead!
\newcommand{\codeline}[1]{\colorbox{codegray}{\lstinline$#1$}}

% define my own colors
\definecolor{codegray}{gray}{0.97}
\definecolor{stringgreen}{rgb}{0.0, 0.7, 0.6} % TODO modify this if you have not already done so

% meta-information
\newcommand{\topic}{
	% TODO fill in the actual topic
	Subprozesse in Python
}

% nothing to do here
\title{\topic}
\supertitle{\course}
\date{\today}

% the actual document
\begin{document}

\maketitle

\begin{frame}
	\tableofcontents
\end{frame}


\section{Grundlagen}
\subsection{Das Modul subprocess}
\begin{frame}[fragile]{Das Modul subprocess}
	Das Modul \texttt{subprocess} erlaubt die Ausf\"uhrung externer Befehle und
	Skripte von einem Python Skript aus. Man kann sich die Funktionsweise \"ahnlich
	der eines Terminals vorstellen.
\end{frame}


\subsection{Eigenschaften}
\begin{frame}[fragile]{Eigenschaften}
	\begin{itemize}
		\item Subprozesse laufen asynchron
		\item Sie laufen direkt auf dem System, nicht in einer Shell
		(wenn nicht anders festgelegt)
		\item Verf\"ugbare Programme h\"angen vom System ab, auf dem sie ausgef\"uhrt werden
		\item Der Aufruf ist allgemeing\"ultig, die Bibliothek verwandelt den
		Aufruf unter Windows in einen kompatiblen \texttt{CreateProcess()} String
	\end{itemize}
\end{frame}


\section{Konstanten}
\subsection{File Descriptoren}
\begin{frame}{File Descriptoren}
	\begin{description}
		\item[DEVNULL] Der systeminterne 'M\"ulleimer'
		\item[PIPE] Die Verbindung zwischen zwei Prozessen
		\item[STDOUT] Die Standardausgabe oder der laufende Prozess
	\end{description}
\end{frame}

\subsection{Exceptions}
\begin{frame}[fragile]{Exceptions}
	\begin{description}
		\item[\texttt{SubprocessError}] Der Standardfehler dieses Moduls
		\item[\texttt{TimeoutError}] Ein Timeout ist aufgetreten
		\item[\texttt{CalledProcessError}] Der Subprozess endete auf eine unerwartete Art
	\end{description}
\end{frame}


\section{Popen Klasse}
\begin{frame}[fragile]{Die Popen Klasse}
	Die \texttt{Popen} Klasse bildet die Basis des Moduls \texttt{subprocess}. \\
	Die Funktionssignatur sieht wie folgt aus:
	\lstinputlisting{resources/08_process/popen_call.py}
\end{frame}

\subsection{Wichtige Argumente}
\begin{frame}[fragile]{Einige wichtige Argumente}
	\begin{description}
		\item[\texttt{args}] Die aufzurufenden Argumente. Sollten vom Typ
			\texttt{tuple} oder \texttt{list} sein. Im Prinzip splittet man den
			Konsolenbefehl an den Leerzeichen: \\
			\texttt{ls -A *.md} entspricht \texttt{['ls', '-A', '*.md']}
		\item[\texttt{shell}] F\"uhrt den Befehl in einer Shell aus.
			Sollte \texttt{False} sein (Standard), sonst ist der Aufruf unsicher.
		\item[\texttt{stdout}] zusammen mit \texttt{stdin} und \texttt{stderr} die Input- und
			Output-Verbindungen des Subprozesses \\
			(hier sind \texttt{DEVNULL}, \texttt{PIPE} und \texttt{STDOUT} n\"utzlich)
		\item[\texttt{env}] Umgebungsvariablen des Kindprozesses. Standard ist ein
			Subset von \texttt{os.environ} (dem Python Prozess Environment)
		\item[\texttt{cwd}] Das Arbeitsverzeichnis des Subprozesses
	\end{description}
\end{frame}


\section{Popen Objekte}
\begin{frame}{Popen Objekte}
	Wenn Popen instanziiert wird, wird der darin enthaltene Prozess gestartet und
	das zur\"uckgegebene \texttt{Popen} Objekt enth\"alt Informationen \"uber den
	laufenden Prozess.
\end{frame}

\subsection{Informationen sammeln}
\begin{frame}[fragile]{Informationen sammeln}
	\begin{itemize}
		\item \texttt{process.args} \\
			Gibt die Argumente zur\"uck, mit denen der Prozess aufgerufen wurde.
		\item \texttt{obj.stdout}, \texttt{obj.stdin}, \texttt{obj.stderr} \\
			Input- und Output-Verbindungen, die beim Start gesetzt wurden
		\item \texttt{process.pid} \\
			Vom System zugewiesene \textit{Prozess ID}.
		\item \texttt{process.poll()} \\
			Pr\"uft, ob der Prozess beendet wurde. Gibt den \textit{R\"uckgabewert}
			des Prozesses zur\"uck oder \texttt{None}, wenn der Prozess noch l\"auft.
		\item \texttt{process.returncode} \\
			% TODO
			Der R\"uckgabewert von \texttt{process.poll()}.
	\end{itemize}
\end{frame}

\subsection{Interaktion mit dem Prozess}
\begin{frame}[fragile]{Interaktion mit dem Prozess}
	\begin{itemize}
		\item \texttt{process.wait(timeout=None} \\
			Wartet \textit{timeout} Sekunden auf die Terminierung des Prozesses
			(wartet unendlich lang, wenn timeout \textit{None} ist). \\
			Wirft nach Ablauf von \textit{timeout} eine \texttt{TimeoutExpired} Exception.
		\item \texttt{process.send\_signal(signal)} \\
			Sendet das Signal \textit{signal} an den Prozess (z.B. \texttt{SIGTERM}).
		\item \texttt{process.communicate(input=None, timeout=None)} \\
			Schreibt die Daten aus \textit{input} in den Standardinput (\texttt{stdin})
			des Prozesses (wenn stdin PIPE ist), wartet auf die Terminierung
			des Prozesses und liest die Daten, die der Prozess in \texttt{stdout}
			geschrieben hat (wenn stdout PIPE ist). \textit{timeout} funktioniert wie oben.
	\end{itemize}
\end{frame}

\begin{frame}[fragile]{Interaktion mit dem Prozess}
	\begin{itemize}
		\item \texttt{process.terminate()} \\
			Sendet ein Terminationssignal an den Prozess (\texttt{SIGTERM}).
		\item \texttt{process.kill()} \\
			Erzwingt die Beendigung des Prozesses (\texttt{SIGKILL}).
	\end{itemize}
\end{frame}

\subsection{Nutzung des Kontextmanagers}
\begin{frame}{Popen im Kontextmanager}
	Popen kann mit dem Kontextmanager verwendet werden (siehe \textit{File Handling}). \\
	Der Code daf\"ur w\"urde wie folgt aussehen:
	\lstinputlisting{resources/08_process/context_mgr.py}
\end{frame}

\section{N\"utzliche Funktionen}
\begin{frame}[fragile]{N\"utzliche Funktionen}
	\texttt{subprocess} enth\"alt einige Kurzfassungen f\"ur h\"aufig genutzte Arbeitsabl\"aufe.
	Intern rufen diese allerdings auch nur \texttt{Popen} auf.
\end{frame}

\begin{frame}{call}
	\lstinputlisting[lastline=2]{resources/08_process/defs.py}
	\begin{itemize}
		\item Ruft einen Prozess auf, wartet auf Terminierung
		\item gibt Returncode des Prozesses zur\"uck
		\item \textbf{Beachte:} Einsatz eines unbenannten Aggregators
	\end{itemize}
\end{frame}

\begin{frame}[fragile]{check\_call}
	Argumente entsprechen den Argumenten von \texttt{call()}.
	\begin{itemize}
		\item Ruft Prozess auf, wartet auf Terminierung
		\item gibt nichts zur\"uck, wenn Ausf\"uhrung erfolgreich (Returncode == 0)
		\item wenn Aufruf nicht erfolgreich, wird \texttt{CalledProcessError} geworfen
	\end{itemize}
\end{frame}

\begin{frame}[fragile]{check\_output}
	\lstinputlisting[firstline=4]{resources/08_process/defs.py}
	\begin{itemize}
		\item f\"uhrt Kommando aus und gibt den Prozessoutput zur\"uck
		\item wirft ebenfalls \texttt{CalledProcessError}, wenn der Returncode nicht 0 ist
	\end{itemize}
\end{frame}

% TODO done? place the (sub)section headings correctly, enhance optic
% nothing to do from here on
\end{document}
