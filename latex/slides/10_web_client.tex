\documentclass[ignorenonframetext,]{beamer}
\setbeamertemplate{caption}[numbered]
\setbeamertemplate{caption label separator}{:}
\setbeamercolor{caption name}{fg=normal text.fg}
\usepackage{amssymb,amsmath}
\usepackage{ifxetex,ifluatex}
\usepackage{fixltx2e} % provides \textsubscript
\usepackage{lmodern}
\ifxetex
  \usepackage{fontspec,xltxtra,xunicode}
  \defaultfontfeatures{Mapping=tex-text,Scale=MatchLowercase}
  \newcommand{\euro}{€}
\else
  \ifluatex
    \usepackage{fontspec}
    \defaultfontfeatures{Mapping=tex-text,Scale=MatchLowercase}
    \newcommand{\euro}{€}
  \else
    \usepackage[T1]{fontenc}
    \usepackage[utf8]{inputenc}
      \fi
\fi
% use upquote if available, for straight quotes in verbatim environments
\IfFileExists{upquote.sty}{\usepackage{upquote}}{}
% use microtype if available
\IfFileExists{microtype.sty}{\usepackage{microtype}}{}

% Comment these out if you don't want a slide with just the
% part/section/subsection/subsubsection title:
\AtBeginPart{
  \let\insertpartnumber\relax
  \let\partname\relax
  \frame{\partpage}
}
\AtBeginSection{
  \let\insertsectionnumber\relax
  \let\sectionname\relax
  \frame{\sectionpage}
}
\AtBeginSubsection{
  \let\insertsubsectionnumber\relax
  \let\subsectionname\relax
  \frame{\subsectionpage}
}

\setlength{\parindent}{0pt}
\setlength{\parskip}{6pt plus 2pt minus 1pt}
\setlength{\emergencystretch}{3em}  % prevent overfull lines
\setcounter{secnumdepth}{0}

\title{Python als Webclient}
\date{}

\begin{document}
\frame{\titlepage}

\section{urllib.request}\label{urllib.request}

\begin{frame}

\begin{quote}
The urllib.request module defines functions and classes which help in
opening URLs (mostly HTTP) in a complex world --- basic and digest
authentication, redirections, cookies and more.
\end{quote}

\end{frame}

\begin{frame}[fragile]{urlopen}

\ldots{} öffnet eine URL. (surprise)

\begin{verbatim}
urlopen(url, data=None, [timeout, ]*, cafile=None, capath=None, cadefault=False, context=None)
\end{verbatim}

\textbf{url} kann ein String sein, wie `http://python.org' oder
`ftp://store.stuff.de', oder eine komplexere Anfrage in Form eines
Request objects sein.\\\textbf{data} enthält Daten, die an den Server
gesendet werden. Muss vom Typ \texttt{bytes} oder ein iterable von
\texttt{bytes} Objekten sein.

Bei URLs mit \emph{http}-Requests wird ein
\hyperref[httpresponse-class]{httplib.client.HTTPResponse} Objekt zurück
gegeben.\\Bei \emph{\textbf{ftp}}, \emph{\textbf{file}} und
\emph{\textbf{data}} ein urllib.response.addinfourl Objekt.

\end{frame}

\begin{frame}[fragile]{Request Klasse}

\begin{verbatim}
Request(url, data=None, headers={}, origin_req_host=None, unverifiable=False, method=None)
\end{verbatim}

\end{frame}

\begin{frame}{HTTPResponse Klasse}

\end{frame}

\end{document}
