% The Slide Definitions
%document
\documentclass[10pt]{beamer}
%theme
% \usetheme{metropolis}
% packages
\usepackage{color}
\usepackage{listings}
\usepackage[ngerman]{babel}
\usepackage[utf8]{inputenc}
\usepackage{multicol}


% color definitions
\definecolor{mygreen}{rgb}{0,0.6,0}
\definecolor{mygray}{rgb}{0.5,0.5,0.5}
\definecolor{mymauve}{rgb}{0.58,0,0.82}

\lstset{
    backgroundcolor=\color{white},
    % choose the background color;
    % you must add \usepackage{color} or \usepackage{xcolor}
    basicstyle=\footnotesize\ttfamily,
    % the size of the fonts that are used for the code
    breakatwhitespace=false,
    % sets if automatic breaks should only happen at whitespace
    breaklines=true,                 % sets automatic line breaking
    captionpos=b,                    % sets the caption-position to bottom
    commentstyle=\color{mygreen},    % comment style
    % deletekeywords={...},
    % if you want to delete keywords from the given language
    extendedchars=true,
    % lets you use non-ASCII characters;
    % for 8-bits encodings only, does not work with UTF-8
    frame=single,                    % adds a frame around the code
    keepspaces=true,
    % keeps spaces in text,
    % useful for keeping indentation of code
    % (possibly needs columns=flexible)
    keywordstyle=\color{blue},       % keyword style
    % morekeywords={*,...},
    % if you want to add more keywords to the set
    numbers=left,
    % where to put the line-numbers; possible values are (none, left, right)
    numbersep=5pt,
    % how far the line-numbers are from the code
    numberstyle=\tiny\color{mygray},
    % the style that is used for the line-numbers
    rulecolor=\color{black},
    % if not set, the frame-color may be changed on line-breaks
    % within not-black text (e.g. comments (green here))
    stepnumber=1,
    % the step between two line-numbers.
    % If it's 1, each line will be numbered
    stringstyle=\color{mymauve},     % string literal style
    tabsize=4,                       % sets default tabsize to 4 spaces
    % show the filename of files included with \lstinputlisting;
    % also try caption instead of title
    language = Python,
	showspaces = false,
	showtabs = false,
	showstringspaces = false,
	escapechar = ,
}

\def\ContinueLineNumber{\lstset{firstnumber=last}}
\def\StartLineAt#1{\lstset{firstnumber=#1}}
\let\numberLineAt\StartLineAt



\newcommand{\codeline}[1]{
	\alert{\texttt{#1}}
}


% Author and Course information
% This Document contains the information about this course.

% Authors of the slides
\author{Felix Döring, Felix Wittwer}

% Name of the Course
\institute{Python-Kurs}

% Fancy Logo 
\titlegraphic{\hfill\includegraphics[height=1.25cm]{../templates/fsr_logo_cropped}}



% Custom Bindings
% \newcommand{\codeline}[1]{
%	\alert{\texttt{#1}}
%}


% Presentation title
\title{Grundlagen}
\date{\today}


\begin{document}

\maketitle

\begin{frame}{Gliederung}
	\setbeamertemplate{section in toc}[sections numbered]
	\tableofcontents
\end{frame}

% --------------------------- Dieser Kurs & Links -----------------------------
\section{Dieser Kurs \& Links}
\begin{frame}{Dieser Kurs \& Links}
	\begin{itemize}
		\item Ressourcen
		\begin{itemize}
			\item Beim Suchen: Vermeidet die python 2.7 Doku
			\item \url{docs.python.org}: offizielle Dokumentation
			\item \url{http://auditorium.inf.tu-dresden.de}
			\item \url{https://github.com/fsr}: iFSR auf Github
			\item[ich] \url{mailto:anton.obersteiner1@mailbox.tu-dresden.de}
		\end{itemize}
		% \item Hinweis: SCM's sind hilfreich (\href{https://git-scm.com}{git}, \href{http://mercurial.selenic.com/}{mercurial})
	\end{itemize}
\end{frame}


% ----------------------- Python & Umgebung ------------------------------
\section{Python \& Umgebung}
\begin{frame}{Der Python Interpreter}
	\begin{itemize}
		\item Übliche Python-Versionen: 2.7 und 3.x (bessere Features)
		\item Python auf \url{https://www.python.org} heruntergeladen und installieren
		\item UNIX: \texttt{...install... python3 python3-dev}
		\item<2-> Python wird interpretiert, nicht kompiliert
	\end{itemize}
	\onslide<2->\lstinputlisting{resources/01_getting_started/mistake.py}
\end{frame}
\begin{frame}{Editor/IDE}
	\begin{description}
		\item[Editor] einfache Färbung/Vervollständigung
		\begin{itemize}
			\item \url{https://atom.io} (weil Github)
			\item \url{http://www.sublimetext.com/3}
			\item \url{https://c9.i}: cloud9 (online, free für open source Projekte)
		\end{itemize}
		\item[IDEs] mehr Unterstützung $\to$ besonders für große Projekte
		\begin{itemize}
			\item \href{https://jetbrains.com/pycharm}{PyCharm} (free + professional für Studenten)
			\item IntelliJ, VSCode, ...
			\item \texttt{idle} aus \texttt{python3-dev} oder \texttt{idle}
		\end{itemize}
		\lstinputlisting{resources/01_getting_started/hello_world.py}
		IDLE?
	\end{description}
\end{frame}
\begin{frame}{Python ausführen}
	\begin{itemize}
		\item Python-Datei: \texttt{text\_file.py}
		\item Im Terminal starten: \texttt{python3} bzw. \texttt{Python.exe}
		\item Ausführbar machen: Windows: Python wählen
		\item Linux: Pfad in die erste Zeile!
		\item dann: graphisch oder \texttt{chmod +x Planet.py}
	\end{itemize}
	\lstinputlisting{resources/01_getting_started/hi_dad.py}
	IDLE! $\to$ Variablen, Funktionen, Typen
\end{frame}

% ----------------------- Datentypen ------------------------------
\section{Datentypen}
\begin{frame}{Datentypen}
	\textbf{builtin Datentypen:}\\
	\begin{tabular}{c|l}
		Name & Funktion \\ \hline
		\color{gray}\texttt{object} & \color{gray}Basistyp, alles erbt von \texttt{object} \\
		\onslide<2->{\color{blue}\texttt{str} & \color{blue}Strings, Zeichenketten} \\
		\onslide<3->{\color{blue}\texttt{int} & \color{blue}Ganzzahl "beliebiger" Größe} \\
		\onslide<4->{\texttt{bool} & Wahrheitswert (\texttt{True}, \texttt{False})} \\
		\onslide<5->{\texttt{function} & Funktionen } \\
		\onslide<6->{\color{blue}\texttt{float} & \color{blue}Kommazahl: \texttt{d = sqrt(2); d**2 == 2?} } \\
		\onslide<7->{\texttt{None} & Das Nichts \\
		\color{gray}\texttt{type} & \color{gray}Grundtyp aller Typen (z.B. \texttt{type(x) == int}) \\
		\hline}
		\onslide<8->{\color{blue}\texttt{list} & \color{blue}standard Liste: \texttt{L = [1, 2, "drei"]} } \\
		\onslide<9->{\texttt{tuple} & unveränderbares n-Tupel \\
		\texttt{set} & (mathematische) Menge von Objekten \\
		\color{gray}\texttt{frozenset} & \color{gray}unveränderbare (mathematische) Menge von Objekten \\
		\texttt{dict} & Hash-Map } \\
	\end{tabular}
\end{frame}

% ----------------------- Konvention, Indent & Kommentare ------------------------------
\section{Konvention, Indent \& Kommentare}
\subsection{Namenskonvention}
\begin{frame}[fragile]{Namenskonvention}
	\begin{description}
		\item[\textbf{Klassen}] \textit{PascalCase}
		\item[\textbf{Variablen, Funktionen, Methoden}] \textit{snake\_case} \\
		\item[\textbf{protected Variablen, Funktionen, Methoden}] \alert{\texttt{\_intern}} oder \alert{\texttt{\_\_privat\_\_}}
		\item[\textbf{Merke}] kein Zugriffsmanagement: man könnte, sollte aber nicht
	\end{description}
\end{frame}
\subsection{Indent und Kommentare}
\begin{frame}[fragile]{Indent und Kommentare}
	\begin{itemize}
		\item Funktionen definieren mit \\
			\texttt{def <funktionsname>([parameter\_liste, ...]): block}
		\item Codeblöcke sind gleichmäßig eingerückt (\texttt{Tab} \alert{\texttt{!=}} \texttt{Space}) \\
	\end{itemize}
	\textbf{Kommentare:}
	\lstinputlisting{resources/01_getting_started/comments.py}
\end{frame}

% ------------------------- Aufgabe --------------------------------
\section{Aufgabe}
\begin{frame}[fragile]{Aufgabe}
	Eine simple Text-Funktion: \\[.5cm]
	\lstinputlisting{resources/01_getting_started/present_question.py}
	\onslide<2->\lstinputlisting{resources/01_getting_started/present_concat.py}
	\onslide<3->\lstinputlisting{resources/01_getting_started/present_main.py}
	\onslide<4-> Operatoren?
\end{frame}


% ------------------------------ Operatoren -----------------------------------
\section{Operatoren}
\subsection{gewöhnliche Operatoren}
\begin{frame}[fragile]{Operatoren}
	\begin{description}
		\item[text] \alert{\texttt{+}}, \alert{\texttt{*}}, \alert{\texttt{\%}}
		\item[mathematisch]<2-> \alert{\texttt{+}}, \alert{\texttt{-}}, \alert{\texttt{*}}, \alert{\texttt{/}}, \alert{\texttt{//}}, \alert{\texttt{\%}}, \texttt{2 ** 5} nicht: \texttt{2 \^{} 5}
		\item[bitweise]<2-> \alert{\texttt{\&}}, \alert{\texttt{|}}, \alert{\texttt{<<}}, \alert{\texttt{>>}}, \alert{\texttt{\^}} (xor), \alert{\texttt{\~}} (invertieren)
		\item[vergleichend]<3-> \alert{\texttt{<}}, \alert{\texttt{>}}, \alert{\texttt{<=}}, \alert{\texttt{>=}}, \alert{\texttt{==}} (Wert gleich), \alert{\texttt{is}} (gleiches Objekt/gleiche Referenz)
		\item[logisch]<4-> \alert{\texttt{and}}, \alert{\texttt{or}}, \alert{\texttt{not}}
		\item[Aufgabe]<5-> \texttt{if/else}, \texttt{return <wert>}, \texttt{type}
	\end{description}
	\onslide<5->\lstinputlisting{resources/01_getting_started/if_question.py}
\end{frame}
\subsection{If/Else-Aufgabe}
\begin{frame}[fragile]{If/Else/Boolean}
	\lstinputlisting{resources/01_getting_started/if.py}
	\onslide<2->\lstinputlisting{resources/01_getting_started/if_boolean.py}
\end{frame}
\subsection{besondere Operatoren}
\begin{frame}[fragile]{besondere Operatoren}
	\begin{description}
		\onslide<1->{\item[\alert{\texttt{()}}] für Aufrufbares (Funktionen): \item \texttt{present("{}A", 19, "DD")}}
		\onslide<2->{\item[\alert{\texttt{[]}}] für Datenstrukturen mit Index: \item \texttt{name[0]}, \texttt{"name"[1:3]}}
		\onslide<3->{\item[\alert{\texttt{.}}] für Methoden und Attribute: \item \texttt{"name".index("m")}}
	\end{description}
\end{frame}

% ------------------------ Zusatz und Ausblick ------------------------------
\section{Zusatz und Ausblick}
\begin{frame}[fragile]{Zusatz und Ausblick}
	Unschön:
	\lstinputlisting{resources/01_getting_started/present_concat.py}
	\onslide<2->\lstinputlisting[lastline=10]{resources/01_getting_started/present_format.py}
	\onslide<3->\lstinputlisting[lastline=10]{resources/01_getting_started/present_loop_question.py}
\end{frame}



% nothing to do from here on
\end{document}
